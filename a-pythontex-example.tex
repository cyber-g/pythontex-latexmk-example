\documentclass[12pt,french]{article}
\usepackage[a3paper]{geometry}
\usepackage[utf8]{inputenc}
\usepackage[T1]{fontenc} 
\usepackage{lmodern}
\usepackage{babel}
\usepackage{pythontex}
\setlength{\parindent}{0pt}

\begin{document}

Works on Ubuntu 20.04. \\

Résultat : 

\begin{sympycode}
theta = Symbol (r'\theta')
p = 1/(sqrt(2*pi))*Integral(exp(-theta**2/2),(theta,-oo,oo))
\end{sympycode}
\begin{equation}
    \sympy{p}=\sympy{p.doit()}
\end{equation}

\begin{sympycode}
omega1 = Symbol (r'\omega_1')
omega2 = Symbol (r'\omega_2')
t = Symbol('t')
Phi1 = Symbol (r'\Phi_1')
Phi2 = Symbol (r'\Phi_2')
rfdualtone = cos(omega1*t+Phi1)+cos(omega2*t+Phi2)
\end{sympycode}
\begin{equation}
    \sympy{rfdualtone}=\sympy{rfdualtone.doit()}
\end{equation}

\begin{sympycode}
omegarf = Symbol (r'\omega_{rf}')
deltaomega = Symbol (r'\Delta\omega')
rfdualtonetwo = fu(cos((omegarf+deltaomega)*t+Phi1)+cos((omegarf-deltaomega)*t+Phi2))
\end{sympycode}
\begin{equation}
    \sympy{rfdualtonetwo}=\sympy{rfdualtonetwo.doit()}
\end{equation}
    
\begin{sympycode}
from sympy.simplify.fu import *
from sympy.abc import x
\end{sympycode}
\begin{equation}
    \sympy{TR10i(cos(1)*cos(3) + sin(1)*sin(3))}
\end{equation}

\end{document}